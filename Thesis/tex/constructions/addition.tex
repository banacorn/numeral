\providecommand{\main}{../..}
\documentclass[\main/thesis.tex]{subfiles}
\begin{document}

\section{The Addition Function}\label{addition}

Addition on \lstinline|ℕ| can consequently be defined by recursively ``moving''
the successor function \lstinline|suc| from one number to the another.

\begin{lstlisting}
_+_ : ℕ → ℕ → ℕ
zero  + y = y
suc x + y = suc (x + y)
\end{lstlisting}

We may attemp to define addition on \lstinline|Numeral| with the same strategy.

\begin{lstlisting}
_+_ : ∀ {b d o}
    → Numeral b d o
    → Numeral b d o
    → Numeral b d o
x ∙      + ys = ? x ys
(x ∷ xs) + ys = ? x (xs + ys)
\end{lstlisting}

However, there is not a corresponding successor function on \lstinline|Numeral|
that allows us to recursively move a digit from one numeral to the another with.
What we are lack of is some kind of ``one-sided'' addition function that takes
\textit{a digit} and \textit{a numeral} instead of two numerals.

\begin{lstlisting}
n+ : ∀ {b d o}
    → Digit d
    → Numeral b d o
    → Numeral b d o
\end{lstlisting}

Moreover, to ensure that a system is closed under these operations,
we require these systems to be \lstinline|Continuous|

\begin{lstlisting}
∀ {b d o} → True (Continuous? b d o)
\end{lstlisting}

\subsection{Sum of Two Digits}

Before implementing \lstinline|n+|, we need to take a step back and figure out
how to add \textit{two digits} together.
Adding two digits should result in \textbf{a digit} and
possibly \textbf{a carry} when the sum overflows.

In most of the systems we are familiar with, the carry is a \textit{fixed digit},
often $ 1 $.
However, when it comes to redundant systems, the carry is often an
\textit{unfixed digit}, because there are more than one way of representing
a number.

\subsubsection{View for the Sum}

We capture these caes with a view called \lstinline|Sum|.

\begin{lstlisting}[basicstyle=\ttfamily\scriptsize]
data Sum : (b d o : ℕ) (x y : Digit (suc d)) → Set where
\end{lstlisting}

When the sum of two digits can still be represented with a single digit,
we compute the sum as \lstinline|leftover| and support it with a proof.

\begin{lstlisting}[basicstyle=\ttfamily\scriptsize]
    NoCarry : ∀ {b d o x y}
        → (leftover : Digit (suc d))
        → (property : Digit-toℕ leftover o ≡ sum o x y)
        → Sum b d o x y
\end{lstlisting}

When the sum of two digits exceeds the upper bound of a digit,
we move the exceeding part to \lstinline|carry| and leave the remainder to
\lstinline|leftover|. \lstinline|property| ensures the integrity of this
computation.

We opt for the constructor \lstinline|Fixed| when the exceeding part that goes
to \lstinline|carry| is fixed (equals to \lstinline|1 ⊔ o|) and
\lstinline|Floating| when it is indefinite.

\begin{lstlisting}[basicstyle=\ttfamily\scriptsize]
    Fixed Floating : ∀ {b d o x y}
        → (leftover carry : Digit (suc d))
        → (property : Digit-toℕ leftover o
                        + (Digit-toℕ carry o) * suc b ≡ sum o x y)
        → Sum b d o x y
\end{lstlisting}

We dispense with the implementation of the corresponding view function
\lstinline|sumView| for brevity.
\footnote{about 300 lines of code and reasoning.}

\begin{lstlisting}
sumView : ∀ b d o
    → (¬gapped : (1 ⊔ o) * suc b ≤ suc d)
    → (proper : 2 ≤ suc d + o)
    → (x y : Digit (suc d))
    → Sum b d o x y
\end{lstlisting}

\subsubsection{\lstinline|n+| on Proper Systems}

Implementing \lstinline|n+| on proper systems are fairly simple since most
of the hard work has be done by \lstinline|sumView|.
Carries are added back by recursively calling \lstinline|n+-Proper|.

\begin{lstlisting}[basicstyle=\ttfamily\scriptsize]
n+-Proper : ∀ {b d o}
    → (¬gapped : (1 ⊔ o) * suc b ≤ suc d)
    → (proper : suc d + o ≥ 2)
    → (x : Digit (suc d))
    → (xs : Numeral (suc b) (suc d) o)
    → Numeral (suc b) (suc d) o
n+-Proper {b} {d} {o} ¬gapped proper x xs
    with sumView b d o ¬gapped proper x (lsd xs)
n+-Proper ¬gapped proper x (_ ∙)    | NoCarry leftover property
    = leftover ∙
n+-Proper ¬gapped proper x (_ ∷ xs) | NoCarry leftover property
    = leftover ∷ xs
n+-Proper ¬gapped proper x (_ ∙)    | Fixed leftover carry property
    = leftover ∷ carry ∙
n+-Proper ¬gapped proper x (_ ∷ xs) | Fixed leftover carry property
    = leftover ∷ n+-Proper ¬gapped proper carry xs
n+-Proper ¬gapped proper x (_ ∙)    | Floating leftover carry property
    = leftover ∷ carry ∙
n+-Proper ¬gapped proper x (_ ∷ xs) | Floating leftover carry property
    = leftover ∷ n+-Proper ¬gapped proper carry xs
\end{lstlisting}

\subsubsection{Properties of \lstinline|n+-Proper|}

The property below verifies the correctness of \lstinline|n+-Proper|.

\begin{lstlisting}[basicstyle=\ttfamily\scriptsize]
n+-Proper-toℕ : ∀ {b d o}
    → (¬gapped : (1 ⊔ o) * suc b ≤ suc d)
    → (proper : suc d + o ≥ 2)
    → (x : Digit (suc d))
    → (xs : Numeral (suc b) (suc d) o)
    → ⟦ n+-Proper ¬gapped proper x xs ⟧ ≡ Digit-toℕ x o + ⟦ xs ⟧
n+-Proper-toℕ ¬gapped proper x xs with sumView b d o ¬gapped proper x (lsd xs)
n+-Proper-toℕ ¬gapped proper x (_ ∙)     | NoCarry _ property = property
n+-Proper-toℕ ¬gapped proper x (x' ∷ xs) | NoCarry leftover property =
    begin
        ⟦ leftover ∷ xs ⟧
    ≡⟨ ... property ... ⟩
        Digit-toℕ x o + ⟦ x' ∷ xs ⟧
    ∎
n+-Proper-toℕ ¬gapped proper x (_ ∙)     | Fixed _ _ property = property
n+-Proper-toℕ ¬gapped proper x (x' ∷ xs) | Fixed leftover carry property =
    begin
        ⟦ leftover ∷ n+-Proper ¬gapped proper carry xs ⟧
    ≡⟨ ... n+-Proper-toℕ ... ⟩
        Digit-toℕ leftover o + (Digit-toℕ carry o + ⟦ xs ⟧) * suc b
    ≡⟨ ... property ... ⟩
        Digit-toℕ x o + ⟦ x' ∷ xs ⟧
    ∎
n+-Proper-toℕ ¬gapped proper x (_ ∙)     | Floating _ _ property = property
n+-Proper-toℕ ¬gapped proper x (x' ∷ xs) | Floating leftover carry property =
    begin
        ⟦ leftover ∷ n+-Proper ¬gapped proper carry xs ⟧
    ≡⟨ ... n+-Proper-toℕ ... ⟩
        Digit-toℕ leftover o + (Digit-toℕ carry o + ⟦ xs ⟧) * suc b
    ≡⟨ ... property ... ⟩
        Digit-toℕ x o + ⟦ x' ∷ xs ⟧
    ∎
\end{lstlisting}

\subsubsection{Generalize \lstinline|n+-Proper| to all Systems}

By imposing the condition of continuity, we exclude systems that are not
acceptable for \lstinline|n+-Proper|.

\begin{lstlisting}[basicstyle=\ttfamily\scriptsize]
n+ : ∀ {b d o}
    → {cont : True (Continuous? b d o)}
    → (n : Digit d)
    → (xs : Numeral b d o)
    → Numeral b d o
n+ {b} {d} {o}      n xs with numView b d o
n+ {_} {_} {_} {()} n xs | NullBase d o
n+ {_} {_} {_}      n xs | NoDigits b o = NoDigits-explode xs
n+ {_} {_} {_} {()} n xs | AllZeros b
n+ {_} {_} {_}      n xs | Proper b d o proper with Gapped#0? b d o
n+ {_} {_} {_} {()} n xs | Proper b d o proper | yes gapped#0
n+ {_} {_} {_}      n xs | Proper b d o proper | no ¬gapped#0
    = n+-Proper (≤-pred (≰⇒> ¬gapped#0)) proper n xs
\end{lstlisting}

The proof of the correctness of \lstinline|n+-toℕ| also follows the same pattern.

\begin{lstlisting}
n+-toℕ : ∀ {b d o}
    → {cont : True (Continuous? b d o)}
    → (n : Digit d)
    → (xs : Numeral b d o)
    → ⟦ n+ {cont = cont} n xs ⟧ ≡ Digit-toℕ n o + ⟦ xs ⟧
\end{lstlisting}

\subsubsection{Implmentation of the Addition Function}

First, we define the addition function on systems of \lstinline|Proper| with
the one-sided addition function \lstinline|n+-Proper|.

\begin{lstlisting}[basicstyle=\ttfamily\scriptsize]
+-Proper : ∀ {b d o}
    → (¬gapped : (1 ⊔ o) * suc b ≤ suc d)
    → (proper : suc d + o ≥ 2)
    → (xs ys : Numeral (suc b) (suc d) o)
    → Numeral (suc b) (suc d) o
+-Proper ¬gapped proper (x ∙)    ys       = n+-Proper ¬gapped proper x ys
+-Proper ¬gapped proper (x ∷ xs) (y ∙)    = n+-Proper ¬gapped proper y (x ∷ xs)
+-Proper {b} {d} {o} ¬gapped proper (x ∷ xs) (y ∷ ys)
    with sumView b d o ¬gapped proper x y
+-Proper ¬gapped proper (x ∷ xs) (y ∷ ys) | NoCarry leftover property
    = leftover ∷ +-Proper ¬gapped proper xs ys
+-Proper ¬gapped proper (x ∷ xs) (y ∷ ys) | Fixed leftover carry property
    = leftover ∷ n+-Proper ¬gapped proper carry (+-Proper ¬gapped proper xs ys)
+-Proper ¬gapped proper (x ∷ xs) (y ∷ ys) | Floating leftover carry property
    = leftover ∷ n+-Proper ¬gapped proper carry (+-Proper ¬gapped proper xs ys)
\end{lstlisting}

By generalizing \lstinline|+-Proper| to all continuous numeral systems,
the addition function we have been long for can be implemented as follows.

\begin{lstlisting}[basicstyle=\ttfamily\scriptsize]
_+_ : ∀ {b d o}
    → {cont : True (Continuous? b d o)}
    → (xs ys : Numeral b d o)
    → Numeral b d o
_+_ {b} {d} {o}  xs ys with numView b d o
_+_ {cont = ()}   xs ys | NullBase d o
_+_ {cont = cont} xs ys | NoDigits b o = NoDigits-explode xs
_+_ {cont = ()}   xs ys | AllZeros b
_+_ {cont = cont} xs ys | Proper b d o proper with Gapped#0? b d o
_+_ {cont = ()}   xs ys | Proper b d o proper | yes ¬gapped#0
_+_ {cont = cont} xs ys | Proper b d o proper | no ¬gapped#0
    = +-Proper (≤-pred (≰⇒> ¬gapped#0)) proper xs ys
\end{lstlisting}

\end{document}
