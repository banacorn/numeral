\providecommand{\main}{../..}
\documentclass[\main/thesis.tex]{subfiles}
\begin{document}

\section{Implementing Addition with Generalized Successors}\label{addition}

\begin{lstlisting}
data ℕ : Set where
    zero : ℕ
    suc : ℕ → ℕ
\end{lstlisting}

\lstinline|ℕ| has an constructor \lstinline|suc| called ``the successor''
that increments a given number when being applied.
Addition on \lstinline|ℕ| can consequently be defined by recursively ``moving''
\lstinline|suc| from one number to the another.

\begin{lstlisting}
_+_ : ℕ → ℕ → ℕ
zero  + y = y
suc x + y = suc (x + y)
\end{lstlisting}

%
% \begin{lstlisting}
% _+_ : ℕ → ℕ → ℕ
% zero  + y = y
% suc x + y = suc (x + y)
% \end{lstlisting}





% \begin{lstlisting}
% data Numeral : (b d o : ℕ) → Set where
%     _∙  : ∀ {b d o} → Digit d → Numeral b d o
%     _∷_ : ∀ {b d o} → Digit d → Numeral b d o → Numeral b d o
% \end{lstlisting}
%
% The corresponding constructor of \lstinline|suc| on \lstinline|Numeral|
% would be \lstinline|_∷_| since it also increases a given numeral with a digit.
%
%
%  on ``increments'' a number of .


\end{document}
