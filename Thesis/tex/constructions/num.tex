\providecommand{\main}{../..}
\documentclass[\main/thesis.tex]{subfiles}
\begin{document}

\section{Num: a representation for positional numeral systems}\label{num}

\subsection{Definition}

Numerals of positional numeral systems are composed of sequences of digits.
Consequently the definition of {\lstinline|Numeral|} will be similar to that of
{\lstinline|List|},
except that a {\lstinline|Numeral|} must contain at least one digit while a list
may contain no elements at all.
The most significant digit is placed near {\lstinline|_∙|} while the least
significant digit is placed at the end of the sequence.

{\lstinline|Numeral|} has three indices, which corresponds to the three
generalizations we have introduced.

\begin{lstlisting}
infixr 5 _∷_

data Numeral : (b d o : ℕ) → Set where
    _∙  : ∀ {b d o} → Digit d → Numeral b d o
    _∷_ : ∀ {b d o} → Digit d → Numeral b d o → Numeral b d o
\end{lstlisting}

The decimal number ``2016'' for example can be represented as:

\begin{lstlisting}
MMXVI : Numeral 10 10 0
MMXVI = # 6 ∷ # 1 ∷ # 0 ∷ (# 2) ∙
\end{lstlisting}
%
where {\lstinline|#_ : ∀ m {n} {m<n : True (suc m ≤? n)} → Fin n|} converts from
{\lstinline|ℕ|} to {\lstinline|Fin n|} \scmsubst{given}{provided that} the number is small enough.

\lstinline|lsd| \scmsubst{extract}{extracts} the least significant digit of a numeral.

\begin{lstlisting}
lsd : ∀ {b d o} → (xs : Numeral b d o) → Digit d
lsd (x ∙   ) = x
lsd (x ∷ xs) = x
\end{lstlisting}

\subsection{Converting to natural numbers}

Converting to natural numbers is fairly trivial.

\begin{lstlisting}
⟦_⟧ : ∀ {b d o} → (xs : Numeral b d o) → ℕ
⟦_⟧ {_} {_} {o} (x ∙)    = Digit-toℕ x o
⟦_⟧ {b} {_} {o} (x ∷ xs) = Digit-toℕ x o + ⟦ xs ⟧ * b
\end{lstlisting}

\section{Dissecting Numeral Systems with Views}\label{views}

There are many kinds of numeral systems inhabit in {\lstinline|Numeral|}.
These systems have different interesting properties that should be treated
differently, so we sort them into \textbf{four categories} accordingly.

\paragraph{Systems with no digits at all}

The number of digits of a system is determined by the index {\lstinline|d|}.
If {\lstinline|d|} happens to be $ 0 $, then there will be no digits in any of
these systems. Although they seem useless, these systems have plenty of properties.
Since there are not digits at all, any property that is related to digits would
hold vacuously.

\paragraph{Systems with \scmsubst{the base of $ 0 $}{base $0$}}

If {\lstinline|b|}, the base of a system, happens to be $ 0 $,
then only the least significant digit would have effects on the evaluation,
because the rest of the digits would diminish into nothing.

\begin{lstlisting}
⟦ x ∙    ⟧ = Digit-toℕ x o
⟦ x ∷ xs ⟧ = Digit-toℕ x o + ⟦ xs ⟧ * 0
\end{lstlisting}

\paragraph{Systems with only zeros}

Consider when {\lstinline|d|} is set to $ 1 $ and {\lstinline|o|} set to $ 0 $.
There will be one digit. however, the only digit can only be assigned to $ 0 $.

\begin{lstlisting}
0, 00, 000, 0000, ...
\end{lstlisting}

As a result, every numeral would evaluate to $ 0 $ regardless of the base.

\paragraph{``Proper'' systems}

The rest of systems that does not fall into any of the categories above
are considered \textit{proper}.

\subsection{Categorizing Systems}

These ``categories'' are represented with a datatype called {\lstinline|NumView|}
that is indexed by the three indices: {\lstinline|b|}, {\lstinline|d|}, and {\lstinline|o|}.

\begin{lstlisting}
data NumView : (b d o : ℕ) → Set where
    NullBase    : ∀   d o → NumView 0       (suc d) o
    NoDigits    : ∀ b o   → NumView b       0       o
    AllZeros    : ∀ b     → NumView (suc b) 1       0
    Proper      : ∀ b d o → (proper : suc d + o ≥ 2)
                          → NumView (suc b) (suc d) o
\end{lstlisting}

By pattern matching on indices, different configurations of indices are sorted into
different {\lstinline|NumView|}s.

\begin{lstlisting}
numView : ∀ b d o → NumView b d o
numView b       zero          o       = NoDigits b o
numView zero    (suc d)       o       = NullBase d o
numView (suc b) (suc zero)    zero    = AllZeros b
numView (suc b) (suc zero)    (suc o) = Proper b zero (suc o) _
numView (suc b) (suc (suc d)) o       = Proper b (suc d) o _
\end{lstlisting}

Together with \textit{with-abstractions}, we can, for example, define a function
to determine whether a numeral system is interesting or not:

\begin{lstlisting}
interesting : ∀ b d o → Bool
interesting b d o with numView b d o
interesting _ _ _ | NullBase d o         = false
interesting _ _ _ | NoDigits b o         = false
interesting _ _ _ | AllZeros b           = false
interesting _ _ _ | Proper b d o proper  = true
\end{lstlisting}

As we can see, the function {\lstinline|numView|} does more than sorting indices
into different categories. It also reveals relevant information and properties
about these categories. For instance, if a system {\lstinline|Numeral b d o|}
is classified as \textit{Proper}, then we know that:

\begin{itemize}
    \item {\lstinline|b|} is greater than $ 0 $.
    \item {\lstinline|d|} is also greater than $ 0 $.
    \item {\lstinline|o|} can be any value as long as {\lstinline|d + o ≥ 2|};
        we name this requirement \textit{proper}.
\end{itemize}

\subsection{Views}

The sole purpose of {\lstinline|NumView|} is to sort out and expose some
interesting properties about its indices.
Such datatypes are called \textit{views}\cite{wadler1987views} as they present
different aspects of the same object.
Functions like {\lstinline|numView|} are called \textit{view functions} or
\textit{eliminators}\cite{mcbride2004views} because they provide different ways
of eliminating a datatype.

Views are \textbf{reusable} as they free us from having to pattern match on the
same indices or data again and again. On the other hand, they can be customized
to our needs since they are just \textit{ordinary functions}.
We will define more views and use them extensively in the coming sections.

\section{Properties of each Category}

\paragraph{NoDigits}

Although systems with no digits have no practical use,
they are pretty easy to deal with because all properties related to digits would
hold unconditionally for systems of {\lstinline|NoDigits|}.
This is proven by deploying \textit{the principle of explosion}.

\begin{lstlisting}
NoDigits-explode : ∀ {b o a} {Whatever : Set a}
    → (xs : Numeral b 0 o)
    → Whatever
NoDigits-explode (() ∙   )
NoDigits-explode (() ∷ xs)
\end{lstlisting}

\paragraph{NullBase}

The theorem below states that, evaluating a numeral of {\lstinline|NullBase|}
would results the same value as evaluating its least significant digit.

\begin{lstlisting}
toℕ-NullBase : ∀ {d o}
    → (x : Digit d)
    → (xs : Numeral 0 d o)
    → ⟦ x ∷ xs ⟧ ≡ Digit-toℕ x o
toℕ-NullBase {d} {o} x xs =
    begin
        Digit-toℕ x o + ⟦ xs ⟧ * 0
    ≡⟨ cong (λ w → Digit-toℕ x o + w) (*-right-zero ⟦ xs ⟧) ⟩
        Digit-toℕ x o + 0
    ≡⟨ +-right-identity (Digit-toℕ x o) ⟩
        Digit-toℕ x o
    ∎
\end{lstlisting}

\paragraph{AllZeros}

The theorem below states every numeral of {\lstinline|AllZeros|}
would evaluate to $ 0 $ regardless of the base.
We pattern match on the digit to eliminate other possible cases to exploit the
fact that there is only one digit in such numerals.

\begin{lstlisting}
toℕ-AllZeros : ∀ {b} → (xs : Numeral b 1 0) → ⟦ xs ⟧ ≡ 0
toℕ-AllZeros     (z    ∙   ) = refl
toℕ-AllZeros     (s () ∙   )
toℕ-AllZeros {b} (z    ∷ xs)
    = cong (λ w → w * b) (toℕ-AllZeros xs)
toℕ-AllZeros     (s () ∷ xs)
\end{lstlisting}

\end{document}
