\providecommand{\main}{../..}
\documentclass[\main/thesis.tex]{subfiles}
\begin{document}

\section{Continuous Systems [work in process]}\label{continuous}

We say a numeral system is \textit{continuous} if there are no gaps in the number
line and we can always find a successor after each numeral.
In other words, a system is continuous if every numeral is \textit{incrementable}.

\begin{lstlisting}
Continuous : ∀ b d o → Set
Continuous b d o = (xs : Numeral b d o) → Incrementable xs
\end{lstlisting}

Bounded systems are deemed to be incontinuous because there are no successor
after the maximum numeral.

\begin{lstlisting}
Bounded⇒¬Continuous : ∀ {b d o}
    → Bounded b d o
    → ¬ (Continuous b d o)
Bounded⇒¬Continuous (xs , max) claim
    = contradiction (claim xs) (Maximum⇒¬Incrementable xs max)
\end{lstlisting}

Since systems of \lstinline|NullBase| and \lstinline|AllZeros| are all bounded,
they cannot be continuous. We will only be concerning ourselves with systems of
\lstinline|Proper| in the rest of the section.

It is also obvious that systems possessing numerals of \lstinline|GappedEndpoint|
must be incontinuous because their number lines are literally gapped.
But the problem is, how do we know if a system is gapped?

% \subsection{Observations}
%
% \lstinline|Numeral 4 3 0| is a quaternary (base-4) system with only 3 digits:
% $0, 1, 2$.
% Suppose we plot all values of numerals of \lstinline|Numeral 4 3 0| onto a number
% line, it would have a series of gaps that are ever widening as shown in the figure below.
%
% \begin{center}
%     \begin{adjustbox}{max width=\textwidth}
%         \begin{tikzpicture}[spy using outlines]
%
%             \foreach \j in {0,...,2} {
%                 \foreach \i in {0,...,2} {
%                     \draw[fill=black] ({\j*4 + \i}, 0.5) rectangle ({\j*4 + \i + 0.75}, 0.75);
%                 };
%             };
%
%             \draw[ultra thick] (0, 0) -- (16, 0);
%
%             % ticks
%             \foreach \i in {0, ..., 64} {
%                 \draw[thick] ({\i*0.25},0) -- ({\i*0.25},-0.2);
%             };
%             \foreach \i in {0, ..., 16} {
%                 \pgfmathsetmacro{\j}{int(\i * 4)}
%                 \draw[thick] (\i,0) -- (\i,-0.3)
%                     node[below, scale=0.8] {\j};
%             };
%
%             % spies
%             \spy[rectangle,lens={scale=8}, size=5cm, connect spies]
%                 on (0.875,0.75) in node [left] at (5.25,5);
%             \spy[rectangle,lens={scale=2}, size=5cm, connect spies]
%                 on (3.375,0.75) in node [left] at (10.5,5);
%
%             % gaps
%             \draw[ultra thick, loosely dotted] (1.75,5) -- (1.75,6);
%             \draw[ultra thick, loosely dotted] (3.75,5) -- (3.75,6);
%             \draw[ultra thick, decoration={brace,mirror},decorate]
%                 (3.75, 6) -- (1.75, 6);
%             \node at (2.75, 6.5) {$1$};
%             \draw[ultra thick, loosely dotted] (6.75,5) -- (6.75,6);
%             \draw[ultra thick, loosely dotted] (9.25,5) -- (9.25,6);
%             \draw[ultra thick, decoration={brace,mirror},decorate]
%                 (9.25, 6) -- (6.75,6);
%             \node at (8, 6.5) {$5$};
%             \draw[ultra thick, loosely dotted] (10.75, 0.5) -- (10.75,6);
%             \draw[ultra thick, loosely dotted] (16,0.5) -- (16,6);
%             \draw[ultra thick, decoration={brace,mirror},decorate]
%                 (16, 6) -- (10.75,6);
%             \node at (13.375, 6.5) {$21$};
%
%         \end{tikzpicture}
%     \end{adjustbox}
% \end{center}
%
% \subsection{The Relation between Gaps}
%
% In the section~\ref{next}, we have propositions for describing these gaps.
%
% \begin{lstlisting}
% Gapped#0 : ∀ b d o → Set
% Gapped#0 b d o = suc d < carry o * suc b
%
% Gapped#N : ∀ b d o
%     → (xs : Numeral (suc b) (suc d) o)
%     → (proper : 2 ≤ suc (d + o))
%     → Set
% Gapped#N b d o xs proper
%     = suc d < (⟦ next-numeral-Proper xs proper ⟧ ∸ ⟦ xs ⟧) * suc b
%
% Gapped : ∀ {b d o}
%     → (xs : Numeral (suc b) (suc d) o)
%     → (proper : 2 ≤ suc (d + o))
%     → Set
% Gapped {b} {d} {o} (x ∙)    proper = Gapped#0 b d o
% Gapped {b} {d} {o} (x ∷ xs) proper = Gapped#N b d o xs proper
% \end{lstlisting}
%
% \lstinline|Gapped#0| only describes the first gap of a system, whereas the rest
% of the gaps are covered by \lstinline|Gapped#N|.
% Compared to \lstinline|Gapped#N|, the definition of \lstinline|Gapped#0| is much
% less demanding because it only depends on the three indices, making it really
% easy to be determined.


\end{document}
