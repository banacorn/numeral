\providecommand{\main}{../..}
\documentclass[\main/thesis.tex]{subfiles}
\begin{document}

\subsection{Continuous Systems}\label{continuous}

We say a numeral system is \textit{continuous} if there are no gaps in the number
line and we can always find a successor after each numeral.
In other words, a system is continuous is every numeral is \textit{incrementable}.

\begin{lstlisting}
Continuous : ∀ b d o → Set
Continuous b d o = (xs : Numeral b d o) → Incrementable xs
\end{lstlisting}

Bounded systems are deemed to be incontinuous because there are no successor
after the maximum numeral.

\begin{lstlisting}
Bounded⇒¬Continuous : ∀ {b d o}
    → Bounded b d o
    → ¬ (Continuous b d o)
Bounded⇒¬Continuous (xs , max) claim
    = contradiction (claim xs) (Maximum⇒¬Incrementable xs max)
\end{lstlisting}

It is also obvious that systems possessing numerals of \lstinline|GappedEndpoint|
must be incontinuous because their number lines are literally gapped.
But the problem is, how do we know which system is gapped?

Let's do some observation.
Suppose we plot all numerals of \lstinline|Numeral 4 3 0| onto a number line.
This is a quaternary (base-4) system with only 3 digits: $0, 1, 2$.

\begin{center}
    \begin{adjustbox}{max width=\textwidth}
        \begin{tikzpicture}
            [
                %using the 'spy' to magnify a part of the picture
                spy using outlines={rectangle,lens={scale=3}, size=8cm, connect spies}
            ]
            % the frame
            % \path[clip] (5.5, -3) rectangle (17.5, 1.5);

            \foreach \k in {0,...,2} {
                \foreach \j in {0,...,2} {
                    \foreach \i in {0,...,2} {
                        \draw[fill=black] ({\k*4 + \j + \i*0.25}, 0.5) rectangle ({\k*4 + \j + \i*0.25 + 0.1875}, 0.7);
                    };
                };
            };

            \draw[ultra thick] (0, 0) -- (16, 0);

            % ticks
            \draw[thick] (0,0) -- (0,-0.1)
                node[below, scale=0.8] {0};
            \foreach \i in {1,...,9} {
                \pgfmathsetmacro{\j}{int(\i * 10)}
                \draw[thick] ({\i*0.625},0) -- ({\i*0.625},-0.1)
                    node[below, scale=0.5] {\j};
            };
            \draw[thick] (6.25,0) -- (6.25,-0.1)
                node[below, scale=0.8] {100};
            \foreach \i in {11,...,19} {
                \pgfmathsetmacro{\j}{int(\i * 10)}
                \draw[thick] ({\i*0.625},0) -- ({\i*0.625},-0.1)
                    node[below, scale=0.5] {\j};
            };
            \draw[thick] (12.5,0) -- (12.5,-0.1)
                node[below, scale=0.8] {200};
            \foreach \i in {21,...,25} {
                \pgfmathsetmacro{\j}{int(\i * 10)}
                \draw[thick] ({\i*0.625},0) -- ({\i*0.625},-0.1)
                    node[below, scale=0.5] {\j};
            };

            %   \spy on (0.4,0.4)
            %              in node [left] at (6.5,1);
        \end{tikzpicture}
    \end{adjustbox}
\end{center}

\end{document}
