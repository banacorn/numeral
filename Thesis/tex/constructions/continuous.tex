\providecommand{\main}{../..}
\documentclass[\main/thesis.tex]{subfiles}
\begin{document}

\subsection{Continuous Systems}\label{continuous}

We say a numeral system is \textit{continuous} if there are no gaps in the number
line and we can always find a successor after each numeral.
In other words, a system is continuous if every numeral is \textit{incrementable}.

\begin{lstlisting}
Continuous : ∀ b d o → Set
Continuous b d o = (xs : Numeral b d o) → Incrementable xs
\end{lstlisting}

Bounded systems are deemed to be incontinuous because there are no successor
after the maximum numeral.

\begin{lstlisting}
Bounded⇒¬Continuous : ∀ {b d o}
    → Bounded b d o
    → ¬ (Continuous b d o)
Bounded⇒¬Continuous (xs , max) claim
    = contradiction (claim xs) (Maximum⇒¬Incrementable xs max)
\end{lstlisting}

It is also obvious that systems possessing numerals of \lstinline|GappedEndpoint|
must be incontinuous because their number lines are literally gapped.






But the problem is, how do we know which system is gapped?

\subsubsection{Observations}

\lstinline|Numeral 4 3 0| is a quaternary (base-4) system with only 3 digits:
$0, 1, 2$.
Suppose we plot all values of numerals of \lstinline|Numeral 4 3 0| onto a number
line, it would have a series of gaps that are ever widening.

\begin{center}
    \begin{adjustbox}{max width=\textwidth}
        \begin{tikzpicture}[spy using outlines]

            \foreach \j in {0,...,2} {
                \foreach \i in {0,...,2} {
                    \draw[fill=black] ({\j*4 + \i}, 0.5) rectangle ({\j*4 + \i + 0.75}, 0.75);
                };
            };

            \draw[ultra thick] (0, 0) -- (16, 0);

            % ticks
            \foreach \i in {0, ..., 64} {
                \draw[thick] ({\i*0.25},0) -- ({\i*0.25},-0.2);
            };
            \foreach \i in {0, ..., 16} {
                \pgfmathsetmacro{\j}{int(\i * 4)}
                \draw[thick] (\i,0) -- (\i,-0.3)
                    node[below, scale=0.8] {\j};
            };

            % spies
            \spy[rectangle,lens={scale=8}, size=5cm, connect spies]
                on (0.875,0.75) in node [left] at (5.25,-5);
            \spy[rectangle,lens={scale=2}, size=5cm, connect spies]
                on (3.375,0.75) in node [left] at (10.5,-5);
            % \spy[rectangle,lens={scale=0.5}, size=5cm, connect spies]
            %     on (13.375,0.75) in node [left] at (15.75,-5);


            % gaps
            \draw[ultra thick, loosely dotted] (1.75,-5) -- (1.75,-4);
            \draw[ultra thick, loosely dotted] (3.75,-5) -- (3.75,-4);
            \draw[ultra thick, decoration={brace,mirror},decorate]
                (3.75, -4) -- (1.75,-4);
            \node at (2.75, -3.5) {$1$};
            \draw[ultra thick, loosely dotted] (6.75,-5) -- (6.75,-4);
            \draw[ultra thick, loosely dotted] (9.25,-5) -- (9.25,-4);
            \draw[ultra thick, decoration={brace,mirror},decorate]
                (9.25, -4) -- (6.75,-4);
            \node at (8, -3.5) {$5$};
            \draw[ultra thick, loosely dotted] (10.75, 0.5) -- (10.75,-3);
            \draw[ultra thick, loosely dotted] (16,0.5) -- (16,-3);
            \draw[ultra thick, decoration={brace,mirror},decorate]
                (10.75, -3) -- (16,-3);
            \node at (13.375, -3.5) {$21$};

        \end{tikzpicture}
    \end{adjustbox}
\end{center}

\end{document}
