\providecommand{\main}{../..}
\documentclass[\main/thesis.tex]{subfiles}
\begin{document}

\section{Systems with an Upper Bound}\label{bounded}

We say that a system is \textit{bounded} if there \textit{exists} a maximum in
such system. In Agda, an existential proposition like this is expressed with
a \textit{dependent sum type}, which is essentially a \textit{pair} where the
type of the second term is dependent on the first.

\begin{lstlisting}
data Σ (A : Set) (B : A → Set) : Set where
    _,_ : (x : A) → B x → Σ A B

proj₁ : ∀ {A B} → Σ A B → A
proj₂ : ∀ {A B} → (pair : Σ A B) → B (proj₁ pair)
\end{lstlisting}

For example, to prove that there exists a right identity for addition on natural numbers;
we place the ``evidence'' on the left and its justification on the right.

\begin{lstlisting}
prop : ∀ n → Σ ℕ (λ e → n + e ≡ n)
prop n = 0 , +-right-identity n
\end{lstlisting}

With a little syntax sugar, we can rewrite the proposition as
{\lstinline|Σ[ x ∈ A ] B|} instead of {\lstinline|Σ A (λ x → B)|}.

\begin{lstlisting}
Bounded : ∀ b d o → Set
Bounded b d o = Σ[ xs ∈ Numeral b d o ] Maximum xs
\end{lstlisting}

The proposition above can be read as follows:
\textit{There exists a numeral {\lstinline|xs|} of type
{\lstinline|Numeral b d o|} such that {\lstinline|Maximum xs|} holds.}
To prove that a system is bounded, we have to place the numeral that we consider
to be a maximum on the left of the pair, and a proof to justify it on the right.

\subsection{Properties of each Category}

\paragraph{NullBase}

We have proven that any numeral with the greatest digit as its LSD is a maximum.

\begin{lstlisting}
Bounded-NullBase : ∀ d o → Bounded 0 (suc d) o
Bounded-NullBase d o =
    (greatest-digit d ∙) ,
    (Maximum-NullBase-Greatest
        (greatest-digit d ∙)
        (greatest-digit-is-the-Greatest d))
\end{lstlisting}

\paragraph{NoDigits}

Since {\lstinline|¬ (Bounded b 0 o)|} reduces to {\lstinline|Bounded b 0 o → ⊥|},
we are given an addition argument, which is a proof claiming that the system is
bounded.
Pattern match on this argument yields a pair with a numeral of
{\lstinline|Numeral b 0 o|} on the left.
At this point, we do not actually care whether the numeral is a maximum
because it should not have existed in the first place.
Hand the numeral to {\lstinline|NoDigits-explode|} and we are done.

\begin{lstlisting}
Bounded-NoDigits : ∀ b o → ¬ (Bounded b 0 o)
Bounded-NoDigits b o (xs , claim) = NoDigits-explode xs
\end{lstlisting}

\paragraph{AllZeros}

Similar to that of {\lstinline|Bounded-NullBase|}:

\begin{lstlisting}
Bounded-AllZeros : ∀ b → Bounded (suc b) 1 0
Bounded-AllZeros b = (z ∙) , Maximum-AllZeros (z ∙)
\end{lstlisting}

\paragraph{Proper}

This proposition is proven by contradicting the fact that
a proper numeral system has no maximum.

\begin{lstlisting}
Bounded-Proper : ∀ b d o → (proper : 2 ≤ suc (d + o))
    → ¬ (Bounded (suc b) (suc d) o)
Bounded-Proper b d o proper (xs , claim) =
    contradiction claim (Maximum-Proper xs proper)
\end{lstlisting}

\subsection{The Decidable Predicate}

We can determine whether a system is bounded by delegating the job to the
helper functions we have defined above.

\begin{lstlisting}
Bounded? : ∀ b d o → Dec (Bounded b d o)
Bounded? b d o with numView b d o
Bounded? _ _ _ | NullBase d o
    = yes (Bounded-NullBase d o)
Bounded? _ _ _ | NoDigits b o
    = no (Bounded-NoDigits b o)
Bounded? _ _ _ | AllZeros b
    = yes (Bounded-AllZeros b)
Bounded? _ _ _ | Proper b d o proper
    = no (Bounded-Proper b d o proper)
\end{lstlisting}

\paragraph{Summary}

\begin{center}
    \begin{adjustbox}{max width=\textwidth}
    \begin{tabular}{ | l | c | c | c | c | }
    \textbf{Properties} & \textbf{NullBase} & \textbf{NoDigits} & \textbf{AllZeros} & \textbf{Proper} \\
    \hline
    has an maximum     & yes & no & yes & no \\
    has an upper bound & yes & no & yes & no \\
    \end{tabular}
    \end{adjustbox}
\end{center}

\end{document}
