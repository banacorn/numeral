\documentclass[../thesis.tex]{subfiles}
\begin{document}
\chapter{Generalizations}\label{generalizations}


\section{Base}

Recall the evaluation function.

$$
    [\![d_0d_1d_2...d_n]\!]_{base}
    =
    \bar{d_0}\times base^0 + \bar{d_1}\times base^1 + \bar{d_2}\times base^2 + ... + \bar{d_n}\times base^n
$$

Where $ \bar{d_n} $ ranges from $ 0 $ to $ base - 1 $ for all $ n $.

As we can see the base of numeral systems has already been generalized.
But nonetheless, it is a good basis for further generalizations.

\section{Offset}

To cooperate unary numerals, we relax the constraint on the range of digit assignment
by introducing a new variable, \textit{offset}:

$$
    [\![d_0d_1d_2...d_n]\!]_{base}
    =
    \bar{d_0}\times base^0 + \bar{d_1}\times base^1 + \bar{d_2}\times base^2 + ... + \bar{d_n}\times base^n
$$

The evaluation of numerals remains the same but the assignment of digits has changed from

$$
    { 0, 1, ..., \textit{offset - 1} }
$$

to

$$
    { \textit{offset}, \textit{offset + 1}, ..., \textit{offset + base - 1} }
$$

The codomain of the digit assignment function is \textit{shifted} by \textit{offset}.
Now that unary numerals would have an offset of $ 1 $
and systems of other bases would have offsets of $ 0 $.

Systems with an offset of $ 1 $ are known as \textit{bijective numerations}
because every number can be represented by exactly one numeral. In other words,
the evaluation function is bijective.

Let us see how to count to ten in a binary numeral system with an offset of $ 1 $.
\footnote{As a reminder, the order of digits are reversed.}

\begin{center}
    \begin{adjustbox}{max width=\textwidth}
    \begin{tabular}{ | l | l |}
    \textbf{Number} & \textbf{Numeral} \\
    \hline
    1 & 1  \\
    2 & 2  \\
    3 & 11 \\
    4 & 21 \\
    5 & 12 \\
    \end{tabular}
    \quad
    \begin{tabular}{ | l | l | }
    \textbf{Number} & \textbf{Numeral} \\
    \hline
    6  & 22 \\
    7  & 111 \\
    8  & 211 \\
    9  & 121 \\
    10 & 221 \\
    \end{tabular}
    \end{adjustbox}
\end{center}

Such a numeral system is also named \textit{1-2 binary system} because its digits
are assigned $ 1 $ and $ 2 $.
Notice that how the symbol of digits are deliberately chosen to match their assigned value.

% \begin{center}
%     \begin{adjustbox}{max width=\textwidth}
%     \begin{tabular}{ | l | r | r | }
%     \textbf{Numeral system} & \textbf{Base} & \textbf{Offset} \\
%     \hline
%     decimal         & 10 & 0 \\
%     binary          & 2  & 0 \\
%     hexadecimal     & 16 & 0 \\
%     unary           & 1  & 1 \\
%     1-2 binary      & 2  & 1 \\
%     \end{tabular}
%     \end{adjustbox}
% \end{center}

A numeral system is said to be \textit{zeroless} if no digits are assigned $ 0 $,
i.e., $ \text{offset} \textgreater 0 $.
Data structures modeled after zeroless systems are called \textit{zeroless representations}.
These containers are preferable to their "zeroful" counterparts.
Because a digit of value $ 0 $ corresponds to a building block with $ 0 $ elements,
and a building block that contains no element is not only useless,
but also hinders traversal as it takes time to skip over these empty nodes.

\section{Number of Digits}

The binary numeral system running in circuits looks different from what we have
in hand.
Surprisingly, these binary numbers can fit into our representation with just a tweak.
If we allow a system to have more digits,
then a fixed-precision binary number can be regarded as a single digit!
To illustrate this,
a 32-bit binary number would become a single digit that ranges from $ 0 $ to $ 2^{32} $,
while everything else including the base remains the same.

Formerly in our representation,
there are exactly \textit{base} number of digits and their assignments range from:

$$
    \text{offset}  ...  \text{offset} + \text{base} - 1
$$

By introducing a new index \textit{\#digit} to generalize the number of digits,
their assignments range from:

$$
    \text{offset}  ...  \text{offset} + \text{\#digit} - 1
$$

The following table contains all of the numeral systems we have addressed so far,
with \textit{base}, \textit{offset}, and \textit{\#digit} taken into account.
\textit{Int32} and \textit{Int64} are respectively 32-bit and 64-bit machine
integers.

\begin{center}
    \begin{adjustbox}{max width=\textwidth}
    \begin{tabular}{ | l | r | r | r | }
    \textbf{Numeral system} & \textbf{Base} & \textbf{\#Digit} & \textbf{Offset} \\
    \hline
    decimal         & 10 & 10 & 0 \\
    binary          & 2  & 2  & 0 \\
    hexadecimal     & 16 & 16 & 0 \\
    unary           & 1  & 1  & 1 \\
    1-2 binary      & 2  & 2  & 1 \\
    Int32           & 2  & $ 2^{32} $ & 0 \\
    Int64           & 2  & $ 2^{64} $ & 0 \\
    \end{tabular}
    \end{adjustbox}
\end{center}



\end{document}
