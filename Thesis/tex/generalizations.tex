\documentclass[../thesis.tex]{subfiles}
\begin{document}
\chapter{Generalizations [still writing]}\label{generalizations}


\section{Base}

Recall the evaluation function.

$$
    [\![d_0d_1d_2...d_n]\!]_{base}
    =
    \bar{d_0}\times base^0 + \bar{d_1}\times base^1 + \bar{d_2}\times base^2 + ... + \bar{d_n}\times base^n
$$

Where $ \bar{d_n} $ ranges from $ 0 $ to $ base - 1 $ for all $ n $.

As we can see the base of numeral systems has already been generalized.
But nonetheless, it is a good start and we will continue to abstract more things
away.

\section{Offset}

To cooperate unary numerals, we relax the constraint on the range of digit
assignment by introducing a new variable, \textit{offset}:

$$
    [\![d_0d_1d_2...d_n]\!]_{base}
    =
    \bar{d_0}\times base^0 + \bar{d_1}\times base^1 + \bar{d_2}\times base^2 + ... + \bar{d_n}\times base^n
$$

The evaluation of numerals remains the same but the assignment of digits has changed from

$$
    { 0, 1, ..., \textit{offset - 1} }
$$

to

$$
    { \textit{offset}, \textit{offset + 1}, ..., \textit{offset + base - 1} }
$$

The codomain of the digit assignment function is \textit{shifted} by \textit{offset}.
Now that unary numerals would have an offset of $ 1 $
and systems of other bases would have offsets of $ 0 $.

\paragraph{1-2 binary system}
Recall \textit{1-2 random access lists} from the previous chapter,
which is the numerical representation of a binary numeral system with an offset
of $ 1 $.
Let us see how to count to ten in the 1-2 binary system.
\footnote{As a reminder, the order of digits are reversed.}

\begin{center}
    \begin{adjustbox}{max width=\textwidth}
    \begin{tabular}{ | l | l |}
    \textbf{Number} & \textbf{Numeral} \\
    \hline
    1 & 1  \\
    2 & 2  \\
    3 & 11 \\
    4 & 21 \\
    5 & 12 \\
    \end{tabular}
    \quad
    \begin{tabular}{ | l | l | }
    \textbf{Number} & \textbf{Numeral} \\
    \hline
    6  & 22 \\
    7  & 111 \\
    8  & 211 \\
    9  & 121 \\
    10 & 221 \\
    \end{tabular}
    \end{adjustbox}
\end{center}

There are no restrictions on the symbols of digits.
But nonetheless, it is reasonable to choose symbols that match their assigned
values, as we choose the symbol ``1'' and ``2'' as digits for the 1-2 binary system.

% \begin{center}
%     \begin{adjustbox}{max width=\textwidth}
%     \begin{tabular}{ | l | r | r | }
%     \textbf{Numeral system} & \textbf{Base} & \textbf{Offset} \\
%     \hline
%     decimal         & 10 & 0 \\
%     binary          & 2  & 0 \\
%     hexadecimal     & 16 & 0 \\
%     unary           & 1  & 1 \\
%     1-2 binary      & 2  & 1 \\
%     \end{tabular}
%     \end{adjustbox}
% \end{center}

\paragraph{bijective numerations}
Systems with an offset of $ 1 $ are also known as \textit{bijective numerations}
because every number can be represented by exactly one numeral. In other words,
the evaluation function is bijective. The 1-2 binary system is one such numeration.

\paragraph{zeroless representations}
A numeral system is said to be \textit{zeroless} if no digits are assigned $ 0 $,
i.e., $ \text{offset} \textgreater 0 $.
Data structures modeled after zeroless systems are called \textit{zeroless representations}.
These containers are preferable to their "zeroful" counterparts.
Because a digit of value $ 0 $ corresponds to a building block with $ 0 $ elements,
and a building block that contains no element is not only useless,
but also hinders traversal as it takes time to skip over these empty nodes,
as we have seen in random access lists from the previous chapter.

\section{Number of Digits}

The binary numeral system running in circuits looks different from what we have
in hand.
Surprisingly, these binary numbers can fit into our representation with just a tweak.
If we allow a system to have more digits,
then a fixed-precision binary number can be regarded as a single digit!
To illustrate this,
a 32-bit binary number would become a single digit that ranges from $ 0 $ to $ 2^{32} $,
while everything else including the base remains the same.

Formerly in our representation,
there are exactly \textit{base} number of digits and their assignments range from:

$$
    \text{offset}  ...  \text{offset} + \text{base} - 1
$$

By introducing a new index \textit{\#digit} to generalize the number of digits,
their assignments range from:

$$
    \text{offset}  ...  \text{offset} + \text{\#digit} - 1
$$



\section{Relation with Natural Numbers}

The following table contains all of the numeral systems we have addressed so far,
with \textit{base}, \textit{offset}, and \textit{\#digit} taken into account.
\footnote{
\textit{Int32} and \textit{Int64} are respectively 32-bit and 64-bit machine
integers.
}

\begin{center}
    \begin{adjustbox}{max width=\textwidth}
    \begin{tabular}{ | l | r | r | r | }
    \textbf{Numeral system} & \textbf{Base} & \textbf{\#Digit} & \textbf{Offset} \\
    \hline
    decimal         & 10 & 10 & 0 \\
    binary          & 2  & 2  & 0 \\
    hexadecimal     & 16 & 16 & 0 \\
    unary           & 1  & 1  & 1 \\
    1-2 binary      & 2  & 2  & 1 \\
    Int32           & 2  & $ 2^{32} $ & 0 \\
    Int64           & 2  & $ 2^{64} $ & 0 \\
    \end{tabular}
    \end{adjustbox}
\end{center}

With great power of generalizations comes poor properties.

Although we are now capable of expressing those numeral systems with just a few
indices, there are also some unexpected inhabitant included in this representation.

\subsection{Surjectiveness}
We can see immediately that,

\end{document}
