\providecommand{\main}{..}
\documentclass[\main/thesis.tex]{subfiles}
\begin{document}
\chapter{Discussion and Conclusion}\label{conclusion}

\section{The Definitinon of Natural Numbers}

Where do natural numbers start from?
This is a commonly debated topic.
Some believe that natural numbers start from $ 1 $,
while others believe that they start from $ 0 $.
Both factions have good reasons to justify their choice.

The same goes for the definition of natural numbers.
Plato, Frege, Russel, and many others all gave their account of the definition
of natural numbers.
Nevertheless, we take the stance of what Benacerraf, an important figure of
\textit{Structuralism}, believes\cite{benacerraf1965numbers}.
It is the abstract structures that numbers represent that is important,
rather than their ``internal'' definitions.

\section{Nil of \lstinline|Numeral|}

The definition of \lstinline|Numeral| started out as a special replica of
\lstinline|List|. That is where we stole the symbol of \lstinline|_∷_|.

\begin{lstlisting}[basicstyle=\ttfamily\scriptsize]
data Numeral : (b d o : ℕ) → Set where
    []  : ∀ {b d o} → Numeral b d o
    _∷_ : ∀ {b d o} → Digit d → Numeral b d o → Numeral b d o
\end{lstlisting}

Soon we face the decision of which value \lstinline|[]| should be assigned to.
$ 0 $ seemed to be a good choice.
However, this would result in a singular point on the number line,
leaving properties such as \lstinline|Continuous| in a quandary.

\begin{center}
    \begin{adjustbox}{max width=\textwidth}
        \begin{tikzpicture}
            % the frame
            \path[clip] (-1, -1) rectangle (11, 2);
            % the spine
            \draw[ultra thick] (0.5,0) -- (10,0);
            % the body

            \draw[ultra thick, fill=white] (0.05, -0.2) rectangle (0.95, +0.2);

            \foreach \i in {3,...,10} {
                \draw[ultra thick, fill=white] ({\i+0.05}, -0.2) rectangle ({\i+0.95}, +0.2);
            };

            % labels
            \draw[->, ultra thick] (0.5,1) -- (0.5,0.5)
                node at (0.5, 1.3) {\lstinline|⟦ [] ⟧| $ = 0 $};
        \end{tikzpicture}
    \end{adjustbox}
\end{center}

Later, we came up with a predicate \lstinline|Null| to prevent \lstinline|[]|
from being evaluated, but we realized that we can do just fine without
\lstinline|[]| shortly after.
This is reason why we insist that every numeral should possess at least one
digit.

\begin{lstlisting}
data Numeral : (b d o : ℕ) → Set where
    _∙  : ∀ {b d o} → Digit d → Numeral b d o
    _∷_ : ∀ {b d o} → Digit d → Numeral b d o → Numeral b d o
\end{lstlisting}

\section{Extending the Universe}

\section{Conclusion}


\end{document}
