\providecommand{\main}{..}
\documentclass[\main/thesis.tex]{subfiles}
\begin{document}

  \begin{center}
  	\LARGE
    \begin{singlespace}
      \textbf{\chineseTitle{}} \\[0.5cm]
    \end{singlespace}

    \begin{singlespace}
    \begin{tabular}{r l}
    	學生     & :\studentCnName{}  \\
        指導教授  & :\advisorCnName{} \hspace{0.1cm} 教授 \\[0.5cm]
    \end{tabular}
    \end{singlespace}

    國立交通大學資訊科學與工程研究所碩士班 \\[0.5cm]
    \makebox[4em][s]{摘要} \\[0.5cm]

  \end{center}
  \normalsize
  \hspace{0.6cm}
    日常生活中充滿了數值,而位值計數系統是最常見的數值表示方式。
    在這篇論文中我們探討並研究各種計數系統的有趣性質與應用,並且在 Agda 裡去建構並且驗證
    我們發展的一種一般化位值計數系統的方法。
    我們不只會去探索我們的計數系統與皮亞諾公理的自然數之間的關係,
    也會發展一套方法去自動轉換彼此之間的定理與證明。
  \\[0.7cm]
  關鍵字:構造性數學、計數系統、定理證明、Agda、一階邏輯
\end{document}
