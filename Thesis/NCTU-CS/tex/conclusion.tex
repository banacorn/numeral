\providecommand{\main}{..}
\documentclass[\main/thesis.tex]{subfiles}
\begin{document}
\chapter{Discussion and Conclusion}\label{conclusion}

\section{The Definitinon of Natural Numbers}

Where do natural numbers start from?
This is a commonly debated topic.
Some believe that natural numbers start from $ 1 $,
while others believe that they start from $ 0 $.
Both factions have good reasons to justify their choice.

The same goes for the definition of natural numbers.
\textit{Plato, Frege, Russel}, and many others all gave their account of the
definition of natural numbers.
Nevertheless, we take the stance of what \textit{Benacerraf},
an important figure of \textit{Structuralism}, believes\cite{benacerraf1965numbers}.
It is the abstract structures that numbers represent that is important,
rather than their ``internal'' definitions.

\section{Nil of \lstinline|Numeral|}

The definition of \lstinline|Numeral| mimics that of \lstinline|List|,
which is where the symbol of \lstinline|_∷_| is stolen from.

\begin{lstlisting}[basicstyle=\ttfamily\scriptsize]
data Numeral : (b d o : ℕ) → Set where
    []  : ∀ {b d o} → Numeral b d o
    _∷_ : ∀ {b d o} → Digit d → Numeral b d o → Numeral b d o
\end{lstlisting}

Soon we face the decision of which value \lstinline|[]| should be assigned to.
$ 0 $ seemed to be a good choice.
However, this would result in a singular point on the number line,
leaving properties such as \lstinline|Continuous| in a quandary.

\begin{figure}[H]
    \centering
    \begin{adjustbox}{max width=\textwidth}
        \begin{tikzpicture}
            % the frame
            \path[clip] (-1, -1) rectangle (11, 2);
            % the spine
            \draw[ultra thick] (0.5,0) -- (10,0);
            % the body

            \draw[ultra thick, fill=white] (0.05, -0.2) rectangle (0.95, +0.2);

            \foreach \i in {3,...,10} {
                \draw[ultra thick, fill=white] ({\i+0.05}, -0.2) rectangle ({\i+0.95}, +0.2);
            };

            % labels
            \draw[->, ultra thick] (0.5,1) -- (0.5,0.5)
                node at (0.5, 1.3) {\lstinline|⟦ [] ⟧| $ = 0 $};
        \end{tikzpicture}
    \end{adjustbox}
\caption{The singularity introduced by the nil}
\label{figure:41}
\end{figure}

Later, we came up with a predicate \lstinline|Null| to prevent \lstinline|[]|
from being evaluated, but we realized that we can do just fine without
\lstinline|[]| shortly after.
This is the reason why we insist that every numeral should possess at least one
digit.

\begin{lstlisting}
data Numeral : (b d o : ℕ) → Set where
    _∙  : ∀ {b d o} → Digit d → Numeral b d o
    _∷_ : ∀ {b d o} → Digit d → Numeral b d o → Numeral b d o
\end{lstlisting}

\section{Numerical Representations and Ornaments}

We started off with numerical representations but then turned to
modeling their corresponding numeral systems.
Perhaps we can go back to index these data structures with our representations,
and see what more can be done.

Moreover, the interesting relationship between numerical representations
and numeral systems can be captured and manipulated with \textit{ornaments}
\cite{mcbride2010ornamental}.
Ornaments allow us to \textit{decorate} plain datatypes with extra information,
or \textit{erase} them from fancy datatypes using a technique called
\textit{induction-recursion}\cite{dybjer1999finite}, which is essentially
a universe construction for coding the datatype itself!

\section{The Universe and Beyond}

In chapter~\ref{translation}, we use universe constructions for expressing and
translating propositions and proofs.
However, our universe is minimalistic and contains only some basic operations
and connectives such as addition, which leaves much to be desired.
We could, for example, support constants in the terms, and perhaps negation in
the predicates.
Maybe we can also construct other arithmetics such as
\textit{Heyting arithmetic} and \textit{Primitive recursive arithmetic}
and see if they work on our representation for numbers in the future.

\section{Conclusion}

We have constructed a generalized positional numeral system
and investigated some of its properties and relation with natural numbers à la
Peano.
We have also demonstrated how to translate propositions and proofs between these
numeral systems.

\end{document}
