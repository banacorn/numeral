
\title{Some title}
\author{
    Someone
}
\date{\today}

\documentclass[12pt, a4paper]{article}
\usepackage{listings}

\begin{document}
\maketitle

\begin{abstract}
This is the paper's abstract \ldots
\end{abstract}

\section{Introduction}

% Numbers are everywhere in our daily lives.

\subsection{What are numbers?}\label{num}

% Argue that what makes certain entity numbers from a structuralist perspective.

\subsection{Positional numeral systems}\label{pos}


\paragraph{Outline}
The remainder of the thesis is organized as follows.
% Section~\ref{agda} gives account of previous work.

\section{A gental introduction to dependently typed programming in Agda}\label{agda}

There are already plenty of tutorials and introductions of Agda\cite{norell2009dependently}.
Nonetheless, we will provide a simple and self-contained tutorial in this section,
covering the part (and only the part) we need in this work.

Some of the more advenced constructions (such as views and universes) used in
the following sections will be introduced along the way.

We assume that all readers have some basic understanding of Haskell, and those
who are familiar with Agda and dependently typed programming may skip this chapter.

\subsection{Some basics}

[introduce some backgrounds of Agda]

\subsection{Dependent types}

\subsection{Simply typed programming in Agda}


Since Agda's syntax is heavily influenced by Haskell, simply typed programming
in Agda is almost the same as in Haskell.

\begin{lstlisting}
\end{lstlisting}




In the beginning there was nothing.

"Let there be data types"



\subsection{Dependently typed programming in Agda}


\section{Representing positional numeral systems}\label{representation}

\subsection{Bases}
\subsection{Offsets}
\subsection{Number of digits}

\section{Properties of Num}
\subsection{Categorizing Num}
\subsection{Views}

\section{Conclusions}\label{conclusions}

\bibliographystyle{abbrv}
\bibliography{thesis}
\end{document}
